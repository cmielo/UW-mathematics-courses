\documentclass[11pt]{article}
\usepackage{amsfonts, amssymb, amsmath, amsthm, mathtools, enumerate, latexsym, setspace, commath}
\usepackage[T1]{fontenc}
\usepackage[a4paper, total={7in, 10.5in}]{geometry}
\pagestyle{empty}

\newcommand{\R}{\mathbb{R}}
\newcommand{\N}{\mathbb{N}}
\newcommand{\Z}{\mathbb{Z}}
\newcommand{\Q}{\mathbb{Q}}
\newcommand{\Prime}{\mathbb{P}}
\newcommand{\Pows}{\mathcal{P}}
\newcommand{\cont}{\mathfrak{c}}

\newcommand{\Dis}{\displaystyle}
\newcommand{\half}{\frac{1}{2}}
\newcommand{\Forall}{{\Large\mbox{$\forall$}}}
\newcommand{\Exists}{{\Large\mbox{$\exists$}}}

\title{Wspólna praca domowa z Analizy 2}
\author{Gracjan Barski, album: 448189}
\date{May 15, 2024}

\begin{document}
\maketitle
\onehalfspacing
\textbf{Rozwiązanie:} \\[5pt]
Zakładam $d \geq 2$. Oznaczmy $S_i = \sup \{ | \partial_i f(x) | \mid x \in D\}$\\ Każde $S_i$ istnieje, ponieważ funkcje pochodne cząstkowe są ograniczone. \\[5pt]
Robimy indukcję po $d$. Dla $d = 2$ mamy:
\begin{align*}
|f(x) - f(y)| &= |f(x_1, x_2) - f(y_1, y_2)| \\ 
&= |f(x_1, x_2) - f(y_1, x_2) + f(y_1, x_2) - f(y_1, y_2)| \\
&\leq |f(x_1, x_2) - f(y_1, x_2)| + |f(y_1, x_2) - f(y_1, y_2)|  
\end{align*}
Z nierówności trójkąta. Teraz z twierdzeniu o wartości średniej mamy, że istnieje $c \in \R$ takie że:
$$\frac{f(x_1, x_2) - f(y_1, x_2)}{x_1 - y_1} = \partial_1 f(c, x_2) \leq S_1$$
Z tego wnioskujemy
$$f(x_1, x_2) - f(y_1, x_2) \leq S_1 |x_1 - y_1|$$
Analogicznie dla $S_2$. Więc mamy:
$$|f(x) - f(y)| \leq S_1 |x_1 - y_1| + S_2 |x_2 - y_2|$$
Co z nierówności Cauchy'ego-Schwarza daje 
$$|f(x) - f(y)| \leq \sqrt{S_1^2 + S_2^2} \cdot \sqrt{(x_1 - y_1)^2 + (x_2 - y_2)^2} = \sqrt{S_1^2 + S_2^2} \cdot ||x - y|| = M \cdot ||x - y||$$
Dla pewnego $M \in \R$. \\[5pt]
Hipoteza indukcyjna: Zakładam, że zachodzi:
$$|f(x) - f(y)| = |f(x_1, \ldots, x_k) - f(y_1, \ldots, y_k)| \leq M \cdot ||x - y||$$
Krok: $k \to k + 1$
\begin{align*}
|f(x) - f(y)| &= |f(x_1, \ldots, x_k, x_{k+1}) - f(y_1, \ldots, y_k, y_{k+1})| \\
&= |f(x_1, \ldots, x_k, x_{k+1}) - f(x_1, \ldots, x_k, y_{k+1}) + f(x_1, \ldots, x_k, y_{k+1}) - f(y_1, \ldots, y_k, y_{k+1})| \\
&\leq |f(x_1, \ldots, x_k, x_{k+1}) - f(x_1, \ldots, x_k, y_{k+1})| + |f(x_1, \ldots, x_k, y_{k+1}) - f(y_1, \ldots, y_k, y_{k+1})|
\end{align*}
$|f(x_1, \ldots, x_k, x_{k+1}) - f(x_1, \ldots, x_k, y_{k+1})| \leq \partial_{k+1}f(x_1, \ldots, c) \cdot |x_{k+1} - y_{k+1}| \leq S_{k+1} \cdot |x_{k+1} - y_{k+1}|$ dla pewnego $c \in \R$ (Z twierdzenia o wartości średniej, jak wyżej).
\newpage

$|f(x_1, \ldots, x_k, y_{k+1}) - f(y_1, \ldots, y_k, y_{k+1})|$ możemy traktować jak różnicę funkcji $k$ zmiennych z jednym stałym parametrem. A z założenia indukcyjnego, wiemy, że takie wyrażenie jest ograniczone przez 
$$M \cdot \sqrt{\sum_{i=1}^k |x_i - y_i|^2}$$
Więc mamy 
$$|f(x) - f(y)| \leq S_{k+1} \cdot |x_{k+1} - y_{k+1}| + M \cdot \sqrt{\sum_{i=1}^k |x_i - y_i|^2} \leq \sqrt{S_{k+1}^2 + M^2} \cdot \sqrt{\sum_{i=1}^{k+1} |x_i - y_i|^2} \leq M' \cdot ||x - y||$$
Gdzie druga nierówność wynika znowu z nierówności Cauchy'ego-Schwarza, a $M' \in \R$. \qed 

\end{document}
