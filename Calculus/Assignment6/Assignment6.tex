\documentclass[11pt]{article}
\usepackage{amsfonts, amssymb, amsmath, amsthm, mathtools, enumerate, latexsym, setspace, commath}
\usepackage[T1]{fontenc}
\usepackage[a4paper, total={7in, 10.5in}]{geometry}
\pagestyle{empty}

\newcommand{\R}{\mathbb{R}}
\newcommand{\N}{\mathbb{N}}
\newcommand{\Z}{\mathbb{Z}}
\newcommand{\Q}{\mathbb{Q}}
\newcommand{\Prime}{\mathbb{P}}
\newcommand{\Pows}{\mathcal{P}}
\newcommand{\cont}{\mathfrak{c}}

\newcommand{\Dis}{\displaystyle}
\newcommand{\half}{\frac{1}{2}}
\newcommand{\Forall}{{\Large\mbox{$\forall$}}}
\newcommand{\Exists}{{\Large\mbox{$\exists$}}}

\title{Wspólna praca domowa z Analizy 2}
\author{Gracjan Barski, album: 448189}
\date{23 March, 2024}

\begin{document}
\maketitle
\onehalfspacing
\textbf{Rozwiązanie:} \\[5pt]
Warunek z zadania przekształcamy do postaci:
$$\Exists_{c \in \R} \; \Forall_{x, y \in D} \; \frac{|f(x) - f(y)|}{||x - y||} \leq c$$
Przy założeniu $x \neq y$. Jeśli $x = y$ to mamy $0 \leq 0$ i jest OK. \\[5pt]
Trzeba pokazać, że zbiór
$$X = \left\{ \frac{|f(x) - f(y)|}{||x - y||} : x, y \in D \land x \neq y \right\}$$
jest ograniczony z góry. Wnioskujemy, że jeśli wszystkie pochodne cząstkowe istnieją we wszystkich punktach $D$ i są ograniczone, to $f$ jest ciągła. Ponadto, $D$ jest zbiorem zwartym, więc $f$ jest również ograniczona. Więc dla każdego $x, y \in D$, $|f(x) - f(y)|$ jest wartością skończoną, tak samo jak $||x - y||$. \\[5pt]
Teraz wystarczy rozważyć co się dzieje, jeśli (WLOG) $y \to x$. \\[5pt]
Rozważmy wektor $v = y - x$. Wtedy badane wyrażenie można zapisać:
$$\frac{|f(x + v) - f(x)|}{||v||}$$
Interesuje nas granica
$$\lim_{h \to 0} \frac{|f(x + hv) - f(x)|}{h||v||}$$
Czyli $\partial_{\Vec{v}}f(x)$. Wiemy, że jeśli funkcja posiada w danym punkcie pochodne cząstkowe, to jej pochodna kierunkowa wyraża się wzorem:
$$\nabla f(x) \cdot \frac{\Vec{v}}{||\Vec{v}||}$$
Gdzie $\cdot$ oznacza standardowy iloczyn skalarny. Jako że każda z pochodnych cząstkowych w $\nabla f(x)$ istnieje, to powyższy iloczyn skalarny ma skończoną wartość. Więc pochodne kierunkowe dla każdego $x \in D$ w dowolnym kierunku istnieją i są skończonymi wartościami. Więc istotnie $X$ jest ograniczony, co dowodzi tezy. \qed

\end{document}
