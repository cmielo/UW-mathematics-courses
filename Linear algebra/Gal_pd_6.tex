\documentclass[10pt]{article}
\usepackage{amsfonts, amssymb, amsmath, amsthm, mathtools, enumerate, latexsym, setspace}
\usepackage[T1]{fontenc}
\usepackage[a4paper, total={7in, 10.5in}]{geometry}
\pagestyle{empty}

\newcommand{\R}{\mathbb{R}}
\newcommand{\N}{\mathbb{N}}
\newcommand{\Z}{\mathbb{Z}}
\newcommand{\Q}{\mathbb{Q}}
\newcommand{\Prime}{\mathbb{P}}
\newcommand{\Pows}{\mathcal{P}}
\newcommand{\cont}{\mathfrak{c}}

\newcommand{\Dis}{\displaystyle}
\newcommand{\half}{\frac{1}{2}}
\newcommand{\Forall}{{\Large\mbox{$\forall$}}}
\newcommand{\Exists}{{\Large\mbox{$\exists$}}}

\title{GAL - praca domowa 6. z dnia 12.12.2023}
\author{Gracjan Barski, album: 448189}
\date{\today}

\begin{document}
\maketitle
% \onehalfspacing
\textbf{Rozwiązanie:} \\[10pt]
Mamy odwzorowanie $F \colon \R^{2,3} \to \R[x]_3$ dane wzorem
$$F(A)(x) = [1 \hspace{7pt} x] \cdot A \cdot \begin{bmatrix}
    1 \\
    x \\
    x^2
\end{bmatrix}$$
\begin{enumerate}[(a)]
    \item Pokazać liniowość przekształcenia $F$. Najpierw pokażę addytywność:
    \begin{align}
        F(A + B)(x) &= [1 \hspace{7pt} x] \cdot (A + B) \cdot \begin{bmatrix}
        1 \\
        x \\
        x^2
        \end{bmatrix} \\
        &= [1 \hspace{7pt} x] \cdot A \cdot \begin{bmatrix}
    1 \\
    x \\
    x^2
    \end{bmatrix} + [1 \hspace{7pt} x] \cdot B \cdot \begin{bmatrix}
        1 \\
        x \\
        x^2
    \end{bmatrix} \\
    &= F(A)(x) + F(B)(x)
    \end{align}
    Gdzie równość \textbf{(2)} wynika z rozdzielności mnożenia względem dodawania w pierścieniu macierzy $R^{n,m}$. Więc mamy addytywność przekształcenia. \\[5pt]
    Teraz mnożenie przez skalar z $\R$:
    \begin{align}
        F(\alpha A)(x) &= [1 \hspace{7pt} x] \cdot \alpha A \cdot \begin{bmatrix}
    1 \\
    x \\
    x^2
    \end{bmatrix} \\
    &= \alpha F(A)(x)
    \end{align}
    Gdzie równość \textbf{(5)} wynika z przemienności mnożenia macierzy przez skalary. Więc mamy liniowość przekształcenia $F$.

    \item Znaleźć bazę $\ker A$. Weźmy dowolne $A \in \R^{2,3}$ i zapiszmy:
    $$A = \begin{bmatrix}
        a & b & c \\
        d & e & f
    \end{bmatrix}$$
    Gdzie $a, b, c, d, e, f \in \R$. Teraz chcemy $F(A)(x) = 0$ (0 to zero w pierścieniu wielomianów czyli wielomian zerowy). Mamy:
    \begin{align}
        F(A)(x) &= [1 \hspace{7pt} x] \cdot \begin{bmatrix}
        a & b & c \\
        d & e & f
    \end{bmatrix} \cdot \begin{bmatrix}
    1 \\
    x \\
    x^2
    \end{bmatrix} \\
    &= a + (b+d)x + (c+e)x^2 + fx^3 = 0
    \end{align}
    Gdzie równość \textbf{(6)} jest prawdziwa wtedy i tylko wtedy gdy wszystkie współczynniki otrzymanego wielomianu są zerowe, więc:
    $$a = 0 \land b = -d \land c = -e \land f = 0$$
    Wszystkie macierze spełniające te warunki są w $\ker A$ więc bazą może być na przykład zbiór:
    $$\left\{ \begin{bmatrix}
        0 & 1 & 0 \\
        -1 & 0 & 0
    \end{bmatrix}, \begin{bmatrix}
        0 & 0 & 1 \\
        0 & -1 & 0
    \end{bmatrix} \right\}$$
    
    \newpage
    \item Znaleźć $V$. Weźmy takie $V \subset \R^{2,3}$, którego bazą $V$ jest
    $$\left\{ \begin{bmatrix}
        1 & 0 & 0 \\
        0 & 0 & 0
    \end{bmatrix}, \begin{bmatrix}
        0 & 0 & 0 \\
        1 & 0 & 0
    \end{bmatrix}, \begin{bmatrix}
        0 & 0 & 0 \\
        0 & 1 & 0
    \end{bmatrix}, \begin{bmatrix}
        0 & 0 & 0 \\
        0 & 0 & 1
    \end{bmatrix}\right\}.$$
    Wtedy mamy równość:  
    $$F\mid_V\left(\begin{bmatrix}
    d & 0 & 0 \\
    c & b & a
    \end{bmatrix}\right)(x) = ax^3 + bx^2 + cx + d$$
    Gdzie $a, b, c, d \in \R$ to współczynniki kombinacji liniowej bazy $V$. \\[5pt]
    Trywialnie jest to izomorfizm, ponieważ jeśli zmienimy jakikolwiek współczynnik przy którymś elemencie z bazy, to któryś jego współczynnik będzie inny, bo współczynniki wielomianu są równe współczynnikom kombinacji liniowej macierzy z bazy, a więc będzie to inny wielomian. Więc mamy iniektywność. \\[5pt]
    Jeśli chcemy otrzymać dowolny wielomian to wystarczy jego współczynniki dać jako odpowiednie współczynniki w kombinacji liniowej bazy $V$, więc mamy surjektywność. A z tego mamy izomorfizm. \qed
\end{enumerate}
\end{document}
