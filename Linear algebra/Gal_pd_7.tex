\documentclass[11pt]{article}
\usepackage{amsfonts, amssymb, amsmath, amsthm, mathtools, enumerate, latexsym, setspace}
\usepackage[T1]{fontenc}
\usepackage[a4paper, total={7.5in, 10.5in}]{geometry}
\pagestyle{empty}

\newcommand{\R}{\mathbb{R}}
\newcommand{\N}{\mathbb{N}}
\newcommand{\Z}{\mathbb{Z}}
\newcommand{\Q}{\mathbb{Q}}
\newcommand{\Prime}{\mathbb{P}}
\newcommand{\Pows}{\mathcal{P}}
\newcommand{\cont}{\mathfrak{c}}
\newcommand{\tr}{\mathrm{tr}}

\newcommand{\Dis}{\displaystyle}
\newcommand{\half}{\frac{1}{2}}
\newcommand{\Forall}{{\Large\mbox{$\forall$}}}
\newcommand{\Exists}{{\Large\mbox{$\exists$}}}

\title{GAL - praca domowa 7. z dnia 18.01.2023}
\author{Gracjan Barski, album: 448189}
\date{\today}

\begin{document}
\maketitle
% \onehalfspacing
\textbf{Rozwiązanie:}
\begin{enumerate}[(a)]
    \item Pokazać, że poniższa funkcja jest iloczynem skalarnym na przestrzeni liniowej $\R^{n,n}$ nad $\R$:
    $$\langle A, B \rangle = \tr(A^TB)$$
    Weźmy dowolne $X, Y, Z \in \R^{n,n}$ oraz $\alpha_1, \alpha_2 \in \R$ \\
    Aby funkcja była iloczynem skalarnym musi mieć następujące własności:
    \begin{enumerate}[I.]
        \item $\langle X, \alpha_1 Y + \alpha_2 Z  \rangle = \alpha_1 \langle X, Y \rangle + \alpha_2 \langle X, Z \rangle$ \\[5pt]
        Teraz pewne przekształcenia:
        \begin{align*}
            \langle X, \alpha_1 Y + \alpha_2 Z  \rangle &= \tr(X^T (\alpha_1 Y + \alpha_2 Z)) \\
            &= \tr(\alpha_1X^TY + \alpha_2 X^TZ) \\
            &= \alpha_1 \langle X, Y \rangle + \alpha_2 \langle X, Z \rangle
        \end{align*}
        Gdzie ostatnie przekształcenie wynika z addytywności i jednorodności operacji śladu macierzy.

        \item $\langle X, Y \rangle = \overline{\langle Y, X \rangle}$ \\[5pt]
        Tak samo jak wyżej:
        \begin{align*}
            \langle X, Y \rangle &= \tr(X^T Y) \\
            &= \tr((Y^T X)^T) \\
            &= \tr(Y^T X) \\
            &= \langle Y, X \rangle \\
            &= \overline{\langle Y, X \rangle}
        \end{align*}
        Gdzie powyższe przekształcenia wynikają z faktu $\tr(A) = \tr(A^T)$ oraz z faktu, że $X,Y \in \R^{n,n}$ więc sprzęganie wyniku nic nie zmienia.

        \item $X \neq 0 \Longrightarrow \langle X, X \rangle > 0$ \\[5pt]
        $X \neq 0$ więc istnieją takie indeksy $1 \leq i, j \leq n$, takie że $x_{i,j} \neq 0$. \\
        Oznaczmy $Y = X^TX$ oraz $y_{i,j}$ jako element w i-tym wierszu i j-tej kolumnie tej macierzy, wtedy mamy: $\langle X, X \rangle = \tr (X^TX) = \tr(Y)$. Rozważmy element $y_{i,i}$:
        $$y_{j, j} = \sum_{k = 1}^n x_{k, j} \cdot x_{k, j} = \sum_{k = 1}^{i - 1} (x_{k, j} \cdot x_{k, j}) + x_{i, j} \cdot x_{i, j} + \sum_{k = i + 1}^{n} (x_{k, j} \cdot x_{k, j}) > 0$$
        Ostatnia nierówność wynika z $x_{i, j} \neq 0$, a wcześniej indeksy przy $x$ są zamienione ponieważ jedna z macierzy była transponowana. Jeśli $y_{j,j} > 0$ to $\tr(Y) > 0$ \qed
    \end{enumerate}

    \item Znaleźć rzut ortogonalny $I_n$ na $\mathrm{span}(B_1, B_2, \ldots, B_n)$ \\[10pt]
    Oznaczmy $B = \mathrm{span}(B_1, B_2, \ldots, B_n)$ Pokażmy, że zbiór $\{B_1, B_2, \ldots, B_n\}$ jest bazą ortogonalną $B$. Rozważmy kombinację liniową:
    \begin{align*}
    \alpha_1 B_1 + \alpha_2 B_2 + \ldots + \alpha_n B_n &= 0 \\
    \begin{bmatrix}
        \alpha_1 & \alpha_2 & \cdots & \alpha_n \\
        \alpha_1 & \alpha_2 & \cdots & \alpha_n \\
        \vdots & \vdots & \ddots & \vdots \\
        \alpha_1 & \alpha_2 & \cdots & \alpha_n
    \end{bmatrix} &= 0
    \end{align*}
    Z tego wnioskujemy, że $\alpha_1 = \alpha_2 = \ldots = \alpha_n = 0$, więc układ jest niezależny liniowo, więc jest bazą podprzestrzeni $B$.\\[5pt]
    Teraz pokażmy, że dla dowolnych $1 \leq i, j \leq n$, takich że $i \neq j$ zachodzi $B_i \perp B_j$. Rozważmy $\langle B_i, B_j \rangle = \tr (B_i^T B_j)$ Chcielibyśmy, aby to było równe 0. Więc suma wyrazów na diagonali musi być równa 0, jednak jak się okaże, każdy wyraz na diagonali jest równy 0. Oznaczmy $X = B_i^T B_j$. Rozpatrzmy dla dowolnego $1 \leq i \leq n$ wyraz $x_{i,i}$:
    $$x_{i,i} = \sum_{k = 1}^n b_{k, i} \cdot b_{k, i} = 0$$
    Gdzie ostatnia równość wynika z faktu, że macierze $B_i, B_j$ mają jedynki w innych kolumnach więc nigdy na siebie nie trafią. Więc mamy bazę ortogonalną. Teraz wiemy, że rzut $I_n$ na $B$ jest wyrażony wzorem:
    $$P_B(I_n) = \sum_{j = 1}^n \frac{\langle B_j, I_n \rangle}{\langle B_j, B_j \rangle} \cdot B_j$$
    Dla dowolnego $1 \leq j \leq n$ Rozważmy: $\langle B_j, I_n \rangle = \tr(B_j^T I_n) = 1$ co wynika ze struktury macierzy $B_j$ i $I_n$. \\[5pt]
    Dla dowolnego $1 \leq j \leq n$ Rozważmy: $\langle B_j, B_j \rangle = \tr(B_j^T B_j) = n$ co wynika ze struktury macierzy $B_j$.\\[5pt]
    Więc otrzymujemy:
    $$P_B(I_n) = \sum_{k = 1}^n \frac{1}{n}B_j$$
    Czyli rzut $I_n$ na $B$ to
    $$\frac{1}{n} \begin{bmatrix}
        1 & 1 & \cdots & 1 \\
        1 & 1 & \cdots & 1 \\
        \vdots & \vdots & \ddots & \vdots \\
        1 & 1 & \cdots & 1
    \end{bmatrix}$$
\end{enumerate}

\end{document}
