\documentclass[11pt]{article}
\usepackage{amsfonts, amssymb, amsmath, amsthm, mathtools, enumerate, latexsym, setspace}
\usepackage[T1]{fontenc}
\usepackage[a4paper, total={7in, 10.5in}]{geometry}
\pagestyle{empty}

\newcommand{\R}{\mathbb{R}}
\newcommand{\N}{\mathbb{N}}
\newcommand{\Z}{\mathbb{Z}}
\newcommand{\Q}{\mathbb{Q}}
\newcommand{\Prime}{\mathbb{P}}
\newcommand{\Pows}{\mathcal{P}}
\newcommand{\cont}{\mathfrak{c}}
\newcommand{\tr}{\mathrm{tr}}

\newcommand{\Dis}{\displaystyle}
\newcommand{\half}{\frac{1}{2}}
\newcommand{\Forall}{{\Large\mbox{$\forall$}}}
\newcommand{\Exists}{{\Large\mbox{$\exists$}}}

\title{GAL - praca domowa 8.}
\author{Gracjan Barski, album: 448189}
\date{\today}

\begin{document}
\maketitle
% \onehalfspacing
\textbf{Rozwiązanie:} \\[10pt]
Najpierw zauważmy: $-A = A^T$, więc jeśli wykorzystamy własność, że dla każdej macierzy $X \in \R^{n,n}$ zachodzi $\det X = \det X^T$, to dostaniemy $\det A = \det -A$. Weźmy $n$ nieparzyste. Wiemy, że przemnożenie macierzy przez $-1$ mnoży jej wyznacznik razy $(-1)^n$, więc mamy $\det A = - \det A$. Więc dostajemy $\det A = 0$. To był przypadek gdy $n$ nieparzyste. \\[10pt]
Teraz rozważmy $n$ parzyste. Przekształćmy wyjściową macierz. Na razie będziemy tylko odejmować jedne wiersze od innych więc nie zmieniamy wyznacznika:
$$
\begin{bmatrix}
0 & -1 & -1 & -1 & \cdots & -1 \\
1 & 0 & -1 & -1 & \cdots & -1 \\
1 & 1 & 0 & -1 & \cdots & -1 \\
1 & 1 & 1 & 0 & \cdots & -1 \\
\vdots & \vdots & \vdots & \vdots & \ddots & \vdots \\
1 & 1 & 1 & 1 & \cdots & 0 \\
\end{bmatrix} 
\overset{\Forall_{2 \leq i \leq n} \; w_i - w_1}{\sim}
\begin{bmatrix}
0 & -1 & -1 & -1 & \cdots & -1 \\
1 & 1 & 0 & 0 & \cdots & 0 \\
1 & 2 & 1 & 0 & \cdots & 0 \\
1 & 2 & 2 & 1 & \cdots & 0 \\
\vdots & \vdots & \vdots & \vdots & \ddots & \vdots \\
1 & 2 & 2 & 2 & \cdots & 1 \\
\end{bmatrix}
$$
Teraz po kolei odejmijmy drugi wiersz od wszystkich poniższych, potem trzeci wierz od wszystkich poniższych, i tak dalej $\ldots$ Otrzymujemy:
$$
\begin{bmatrix}
0 & -1 & -1 & -1 & \cdots & -1 & -1\\
1 & 1 & 0 & 0 & \cdots & 0 & 0\\
0 & 1 & 1 & 0 & \cdots & 0 & 0\\
0 & 0 & 1 & 1 & \cdots & 0 & 0\\
\vdots & \vdots & \vdots & \vdots & \ddots & \vdots & \vdots \\
0 & 0 & 0 & 0 & \cdots & 1 & 0 \\
0 & 0 & 0 & 0 & \cdots & 1 & 1 \\
\end{bmatrix}
$$
Wartość wyznacznika tej macierzy jest taka sama jak wyznacznika macierzy wyjściowej, ponieważ jedynie odejmowaliśmy wiersze od siebie. W pierwszej kolumnie mamy same zera oprócz jedynki w drugim wierszu. Skorzystajmy z rozwinięcia Laplace'a względem pierwszej kolumny, aby policzyć wyznacznik:
$$\det A = -1 \cdot \det 
\begin{bmatrix}
-1 & -1 & -1 & \cdots & -1 & -1\\
1 & 1 & 0 & \cdots & 0 & 0\\
0 & 1 & 1 & \cdots & 0 & 0\\
\vdots & \vdots & \vdots & \ddots & \vdots & \vdots \\
0 & 0 & 0 & \cdots & 1 & 0 \\
0 & 0 & 0 & \cdots & 1 & 1 \\
\end{bmatrix}$$
Warto zaznaczyć, że ta macierz po prawej, której chcemy policzyć wyznacznik, jest rozmiaru $n - 1 \times n - 1$ (oczywiście rozmiar się zmniejszył przez zastosowanie rozwinięcia Laplace'a). Więc jest to macierz o nieparzystej liczbie wierszy i kolumn (ponieważ jesteśmy w przypadku $n$ parzyste). Teraz, jeśli spróbujemy wyzerować pierwszy wiesz, to okaże się, że zostanie nam $-1$ na pierwszym miejscu tego wiersza. Możemy to uzyskać poprzez dodanie ostatniego ($n - 1$-tego) wiersza do pierwszego, potem dodanie $n - 3$-tego wiersza, i tak dalej. Za każdym razem usuwamy dwie $-1$ z pierwszego wiersza, ale ich jest nieparzyście wiele, więc zostanie jedna na początku. Finalnie otrzymamy taką macierz:
$$
\begin{bmatrix}
-1 & 0 & 0 & \cdots & 0 & 0\\
1 & 1 & 0 & \cdots & 0 & 0\\
0 & 1 & 1 & \cdots & 0 & 0\\
\vdots & \vdots & \vdots & \ddots & \vdots & \vdots \\
0 & 0 & 0 & \cdots & 1 & 0 \\
0 & 0 & 0 & \cdots & 1 & 1 \\
\end{bmatrix}
$$
Która jest trójkątna. Oczywiście wyznacznik macierzy trójkątnej jest równy iloczynowi elementów na diagonali, który w tym przypadku wynosi $-1$. Podstawiając do wcześniejszej formuły otrzymujemy:
$$\det A = -1 \cdot -1 = 1$$
W dowodzie zostały użyte macierze "duże" (to znaczy 
$n \geq 5$), więc gdyby ktoś miał wątpliwości, czy ta metoda działa dla macierzy "małych" (to znaczy $2 \leq n \leq 4$), to ten rezultat również sprawdza się dla tych "małych" i można to łatwo sprawdzić ręcznie. \qed 
\end{document}
