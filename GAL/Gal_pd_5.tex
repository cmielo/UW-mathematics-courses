\documentclass[10pt]{article}
\usepackage{amsfonts, amssymb, amsmath, amsthm, mathtools, enumerate, latexsym, setspace}
\usepackage[T1]{fontenc}
\usepackage[a4paper, total={7in, 10.5in}]{geometry}
\pagestyle{empty}

\newcommand{\R}{\mathbb{R}}
\newcommand{\N}{\mathbb{N}}
\newcommand{\Z}{\mathbb{Z}}
\newcommand{\Q}{\mathbb{Q}}
\newcommand{\Prime}{\mathbb{P}}
\newcommand{\Pows}{\mathcal{P}}
\newcommand{\cont}{\mathfrak{c}}

\newcommand{\Dis}{\displaystyle}
\newcommand{\half}{\frac{1}{2}}
\newcommand{\Forall}{{\Large\mbox{$\forall$}}}
\newcommand{\Exists}{{\Large\mbox{$\exists$}}}

\title{GAL - praca domowa 5. z dnia 5.12.2023}
\author{Gracjan Barski, album: 448189}
\date{December 10, 2023}

\begin{document}
\maketitle
\onehalfspacing
\textbf{Rozwiązanie:} \\[10pt]
Mamy macierze $A, B \in R^{n,n}$. Zachodzi $\ker A \cap \, \mathrm{im} \, B = \{0\}$. 
\begin{enumerate}[a)]
    \item $\ker (AB) = \ker B$. Prosty dowód przez wzajemną inkluzję: \\[5pt]
    $(\subseteq)$ Kontrapozycja: Weźmy dowolny wektor $x \notin \ker B$. Wystarczy pokazać $x \notin \ker (AB)$. Jeśli $x \notin \ker B$ to wtedy $Bx \neq 0$. Wiemy też, że $Bx \in \mathrm{im} B$, a z założenia że $\ker A$ i $\mathrm{im} \, B$ mają tylko jeden wspólny element $0$, to otrzymujemy $Bx \notin \ker A$, a wtedy $ABx \neq 0$. Z tego wnioskujemy $x \notin \ker (AB)$. \\[5pt]
    $(\supseteq)$ Weźmy dowolny wektor $x \in \ker B$, wtedy $Bx = 0$. Wiemy że $0 \in \ker A$ (bo 0 należy do jądra każdej macierzy), więc $Bx \in \ker A$, więc $ABx = 0$, a z tego wnioskujemy $x \in \ker (AB)$ \\[5pt]
    Z wzajemnej inkluzji wnioskujemy równość $\ker (AB) = \ker B$.

    \item $\mathrm{im} \; ((AB)^T) = \mathrm{im} \; (B^T)$. Najpierw pokażę inkluzję w prawą stronę:\\[5pt]
    $(\subseteq)$ Przepiszmy inaczej: $\mathrm{im} \; ((AB)^T) = \mathrm{im} \; (B^TA^T)$  Weźmy dowolny wektor $x \in \mathrm{im} \; (B^TA^T)$. Z definicji to oznacza:
    $$\Exists_{y_0 \in \R^n} \; x = B^TA^Ty_0$$
    Teraz oznaczmy $A^Ty_0 = z \in \R^n$ (bo $A \in \R^{n,n}$), więc wyrażenie przyjmuje postać:
    $$x = B^Tz \in \mathrm{im} \; (B^T)$$
    Więc inkluzja udowodniona. \\[5pt]
    Teraz pokażę równość wymiarów tych obrazów macierzy: Z twierdzenia o rzędzie mamy:
    \begin{align*}
        \mathrm{rank} \; B^T + \mathrm{dim \; ker} \; B^T &= n \\
        \mathrm{rank} \; (AB)^T + \mathrm{dim \; ker} \; (AB)^T &= n \end{align*}
    Ponieważ $B^T, (AB)^T \in \R^{n,n}$.
    Z wykładu wiadomo również, że $\mathrm{rank} \; B = \mathrm{rank} \; B^T$. Znowu korzystając z twierdzenia o rzędzie, tym razem dla $B$, otrzymujemy $\mathrm{dim \; ker} \; B = \mathrm{dim \; ker} \; B^T$. \\
    Analogicznie $\mathrm{dim \; ker} \; AB = \mathrm{dim \; ker} \; (AB)^T$. Korzystając z rezultatu z podpunktu a), otrzymujemy:
    $$\mathrm{dim \; ker} \; B^T = \mathrm{dim \; ker} \; B = \mathrm{dim \; ker} \; AB = \mathrm{dim \; ker} \; (AB)^T$$
    Więc istotnie wymiary obrazów są takie same, oraz jeden z nich zawiera się w drugim, a to ze Stwierdzenia 6.13 oznacza, że $\mathrm{im} \; ((AB)^T) = \mathrm{im} \; (B^T)$. \qed
\end{enumerate}
\end{document}
